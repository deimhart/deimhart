\frame{\frametitle{Vorbedingungen für Trac}
\begin{itemize}
	\item Python ab 2.4, 3.0 und grösser wird noch nicht unterstützt. \pause
	\item setuptools \pause
	\item Genshi \pause
	\item Datenbank
	\begin{itemize}
		\item SQLite ist in Python ab Version 2.5 enthalten 
	\end{itemize}
\end{itemize}
}

\frame{\frametitle{Optional für Trac}
\begin{itemize}
	\item Trac kann Standalone ohne weiteren Server laufen \pause
	\item ">richtiger"< Webserver
	\begin{itemize}
		\item Apache 2
		\item andere möglich, aber nicht dokumentiert 
	\end{itemize} \pause
	\item ">richtige"< Datenbank
	\begin{itemize}
		\item MySQL
		\item PostgreSQL 
	\end{itemize} \pause
	\item Versionskontrollsystem
	\begin{itemize}
		\item Subversion
		\item Andere (Bazaar, Git, Mercurial, ...)
	\end{itemize} \pause
\end{itemize}
}

\frame{\frametitle{Installation Trac (1)}
\begin{exampleblock}{Einschränkung für diesen Vortrag}
Wir werden hier Trac mit Apache als Webserver und Subversion als Versionsverwaltungsssystem und Authentifizierung über den Webserver installieren.

Es gibt eine grosse Menge an weiteren Möglichkeiten, die aber hier den Rahmen sprengen würden. \pause
\end{exampleblock}

\begin{block}{Überprüfung der Vorbedingungen}
\begin{itemize}
	\item \texttt{aptitude show apache2}
	\item \texttt{aptitude show libapache2-python}
	\item \texttt{aptitude show python-pysqlite1.1 python-pysqlite2}
	\item \texttt{aptitude show python-setuptools}
\end{itemize}
\end{block}
}
