\documentclass{beamer}		% Texttyp
\usepackage[utf8]{inputenc}		% UTF-8-Zeichensatz verwenden
\usepackage[ngermanb]{babel}	% Deutsche Texttrennung, Überschriften etc.
\usepackage{url}
\urlstyle{tt}

\title[Wikis]{Installation, Parametrisierung und Benutzung von Wikis}	% [optional: Kurzform] Titel der Praesentation
\subtitle[Dokumentieren leicht gemacht]{Dokumentieren leicht gemacht}
\author[]{Roman Hanhart und Dirk Deimeke}	% [optional: Kurzform] Name des Autors
\institute{Ubucon 2010}			% Firma, Unternehmen
\date{16. Oktober 2010}				% Datum (hier: heute)
%\logo{\pgfimage[width=0.9cm,height=0.4cm]{udtheme_logo.png}} % original: 184 x 80 (2,3)
\titlegraphic{\pgfimage[width=2.3cm,height=1cm]{udtheme_logo.png}} % original: 184 x 80 (2,3)
\subject{Wikis}		% Thema
\keywords{wiki, webserver, version control, ubuntu}		% Schluesselwuerter
\beamertemplatenavigationsymbolsempty 		% Abschalten der Navigationselemente
\setbeamertemplate{footline}[frame number]	% Seitenzahlen in Fußzeile einfügen
\setbeamercovered{transparent} % halbtransparente Overlays an. Sollen wir das nutzen?

\usetheme{Dresden}
% \usecolortheme{Wolverine}	% Farbtheme
\usecolortheme{Whale}	% Farbtheme

\begin{document}
\frame{\titlepage}
\frame{\frametitle{Inhalt}\tableofcontents}

\section{Vorstellung} 
\frame{\frametitle{Roman Hanhart}
}
\frame{\frametitle{Roman Hanhart (Ubuntu)}
}

\frame{\frametitle{Dirk Deimeke}
\begin{itemize}
	\item Senior Unix Systems Administrator bei der Credit Suisse in der Schweiz (Best Global Bank laut Euromoney)
	\item Verheiratet, zwei Hunde
	\item Geboren in Wanne-Eickel / Ruhrgebiet / Nordrhein-Westfalen / Deutschland
	\item Bekennender ">Alter Sack"<
	\item Wohnhaft im Grüt / Zürcher Oberland / Kanton Zürich / Schweiz (2008 in die Schweiz ausgewandert)
	\item Fühle mich als Weltbürger und Ruhrie (Ja, auch als Netzbürger)
	\item Mehr unter \url{http://dirk.deimeke.net/} \\
		Podcast \url{http://deimhart.net/}
\end{itemize}
}

\frame{\frametitle{Dirk Deimeke (Ubuntu)}
\begin{itemize}
	\item Vereinsvorsitzender ubuntu Deutschland e. V.
	\item Ansprechpartner für das deutsche LoCo-Team
	\item Ubuntu Member
	\item Organisationsteam der Ubucon
	\item ... mehr unter \url{http://d5e.org/ubuntu}
\end{itemize}
}

\section{Einführung} 
\frame{\frametitle{Einführung}
\begin{block}{Bin ich hier richtig?}
\begin{itemize}
        \item Das wird sich gleich zeigen ... :-) \pause 
\end{itemize}
\end{block}

\begin{block}{Worum geht es?}
\begin{itemize}
        \item Um Wikis im Allgemeinen
	\item Um MediaWiki und Trac im Besonderen \pause 
\end{itemize}
\end{block}

\begin{block}{Worum geht es hier nicht?}
\begin{itemize}
        \item Um eine Marktübersicht vorhandener Wikis
\end{itemize}
\end{block}
}

\frame{\frametitle{Hinweis}
\begin{itemize}
        \item Die Session soll kein sturer Vortrag, sondern vielmehr ein Miteinander sein.
        \item Bei Fragen bitte immer fragen!
\end{itemize}
}

\section{Warum Wiki?} 

\end{document}
