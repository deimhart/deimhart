\documentclass{beamer}
\usepackage[utf8]{inputenc}
\usepackage[ngermanb]{babel}
\usepackage{url}
\urlstyle{tt}

\title[Wikis]{Installation, Parametrisierung und Benutzung von Wikis}
\subtitle[Dokumentieren leicht gemacht]{Dokumentieren leicht gemacht}
\author[]{Roman Hanhart und Dirk Deimeke}
\institute{Ubucon 2010}
\date{16. Oktober 2010}
\titlegraphic{\pgfimage[width=2.3cm,height=1cm]{udtheme_logo.png}}
\subject{Wikis}
\keywords{wiki, webserver, version control, ubuntu}
\beamertemplatenavigationsymbolsempty
\setbeamertemplate{footline}[frame number]
\setbeamercovered{transparent}

\usetheme{Dresden}
\usecolortheme{whale}

\begin{document}

\frame{\titlepage}
\frame{\frametitle{Inhalt}\tableofcontents}

\section{Vorstellung} 
\frame{\frametitle{Roman Hanhart}
\begin{itemize}
        \item Familienvater in Andelfingen im Kanton Zürich
        \item Unser Sohn Ruben ist sieben Jahre jung und nutzt Maverick Meerkat.
        \item Senior System Engineer am Spital Bülach (Zürcher Unterland)
        \item Ich lege Wert auf das Wörtchen \textbf{Senior} ;-) 
        \item verheiratet
        \item Jedi und Hermetiker (philosophischer Möchtegern-Wichtigtuer)
        \item Podcasting: \url{http://deimhart.net} und \url{http://radiotux.de}
        \item mehr über mich: \url{http://tuxhart.ch/kontakt.php}
\end{itemize}

}
\frame{\frametitle{Roman Hanhart (Ubuntu)}
\begin{itemize}
        \item Mitglied im Eidgenössischen LocoTeam \url{https://wiki.ubuntu.com/SwissTeam}
        \item Ikhaya-Team \url{http://wiki.ubuntuusers.de/ubuntuusers/Ikhayateam}
        \item Mitglied im Verein Ubuntu Deutschland \url{http://verein.ubuntu-de.org/}
        \item ab und an an Veranstaltungen rund um Ubuntu zu sehen (Messen, LinuxTag etc.)
        \item Mein Blog zum Thema Ubuntu \url{ubuntublog.ch}
\end{itemize}

}
\frame{\frametitle{Dirk Deimeke}
\begin{itemize}
	\item Senior Unix Systems Administrator bei der Credit Suisse in der Schweiz (Best Global Bank laut Euromoney)
	\item Verheiratet, zwei Hunde
	\item Geboren in Wanne-Eickel / Ruhrgebiet / Nordrhein-Westfalen / Deutschland
	\item Bekennender ">Alter Sack"<
	\item Wohnhaft im Grüt / Zürcher Oberland / Kanton Zürich / Schweiz (2008 in die Schweiz ausgewandert)
	\item Fühle mich als Weltbürger und Ruhrie (Ja, auch als Netzbürger)
	\item Mehr unter \url{http://dirk.deimeke.net/} \\
		Podcast \url{http://deimhart.net/}
\end{itemize}
}

\frame{\frametitle{Dirk Deimeke (Ubuntu)}
\begin{itemize}
	\item Vereinsvorsitzender ubuntu Deutschland e. V.
	\item Ansprechpartner für das deutsche LoCo-Team
	\item Ubuntu Member
	\item Organisationsteam der Ubucon
	\item ... mehr unter \url{http://d5e.org/ubuntu}
\end{itemize}
}

\section{Einführung}
\subsection{Einfuehrung}

\frame{\frametitle{Einführung}
\begin{block}{Bin ich hier richtig?}
\begin{itemize}
        \item Das wird sich gleich zeigen ... :-) \pause 
\end{itemize}
\end{block}

\begin{block}{Worum geht es?}
\begin{itemize}
	\item Um Wikis im Allgemeinen
	\item Um MediaWiki und Trac im Besonderen \pause 
\end{itemize}
\end{block}

\begin{block}{Worum geht es hier nicht?}
\begin{itemize}
        \item Um eine Marktübersicht vorhandener Wikis.
\end{itemize}
\end{block}
}

\frame{\frametitle{Hinweis}
\begin{itemize}
        \item Die Session soll kein sturer Vortrag, sondern vielmehr ein Miteinander sein.
        \item Bei Fragen bitte immer fragen!
\end{itemize}
}

\section{Warum Wiki?} 
% Warum MediaWiki
\subsection{Warum Trac?}

\section{Einrichtung}
\frame{\frametitle{Vorbedingungen}
\begin{itemize}
        \item Webserver
\end{itemize}
}
\frame{\frametitle{Vorbedingungen für Trac}
\begin{itemize}
	\item Python ab 2.4, 3.0 und grösser wird noch nicht unterstützt. \pause
	\item setuptools \pause
	\item Genshi \pause
	\item Datenbank
	\begin{itemize}
		\item SQLite ist in Python ab Version 2.5 enthalten 
	\end{itemize}
\end{itemize}
}

\frame{\frametitle{Optional für Trac}
\begin{itemize}
	\item Trac kann Standalone ohne weiteren Server laufen \pause
	\item ">richtiger"< Webserver \pause
	\begin{itemize}
		\item Apache 2
		\item andere möglich, aber nicht dokumentiert 
	\end{itemize} \pause
	\item ">richtige"< Datenbank
	\begin{itemize}
		\item MySQL
		\item PostgreSQL 
	\end{itemize} \pause
	\item Versionskontrollsystem
	\begin{itemize}
		\item Subversion
		\item Andere (Bazaar, Git, Mercurial, ...)
	\end{itemize} \pause
\end{itemize}
}

\frame{\frametitle{Installation Trac (1)}
\begin{exampleblock}{Einschränkung für diesen Vortrag}
Wir werden hier Trac mit Apache als Webserver und Subversion als Versionsverwaltungsssystem und Authentifizierung über den Webserver installieren.

Es gibt eine grosse Menge an weiteren Möglichkeiten, die aber hier den Rahmen sprengen würden. \pause
\end{exampleblock}

\begin{block}{Überprüfung der Vorbedingungen}
\begin{itemize}
	\item \texttt{aptitude show apache2}
	\item \texttt{aptitude show libapache2-python}
	\item \texttt{aptitude show python-pysqlite1.1 python-pysqlite2}
	\item \texttt{aptitude show python-setuptools}
\end{itemize}
\end{block}
}


\section{Live Demonstration}

\frame{\frametitle{Live Demonstration}
\begin{itemize}
        \item MediaWiki \pause
        \item Trac 
\end{itemize}
}

\end{document}
