\documentclass{beamer}
\usepackage[utf8]{inputenc}
\usepackage[ngermanb]{babel}
\usepackage{url}
\urlstyle{tt}

\title[Wikis]{Installation, Parametrisierung und Benutzung von Wikis}
\subtitle[Dokumentieren leicht gemacht]{Dokumentieren leicht gemacht}
\author[]{Roman Hanhart und Dirk Deimeke}
\institute{Ubucon 2010}
\date{16. Oktober 2010}
\titlegraphic{\pgfimage[width=2.3cm,height=1cm]{udtheme_logo.png}}
\subject{Wikis}
\keywords{wiki, webserver, version control, ubuntu}
\beamertemplatenavigationsymbolsempty
\setbeamertemplate{footline}[frame number]
\setbeamercovered{transparent}

\usetheme{Dresden}
\usecolortheme{whale}

\begin{document}

\frame{\titlepage}
\frame{\frametitle{Inhalt}\tableofcontents}

\section{Vorstellung} 
\frame{\frametitle{Roman Hanhart}
}
\frame{\frametitle{Roman Hanhart (Ubuntu)}
}
\frame{\frametitle{Dirk Deimeke}
\begin{itemize}
	\item Verheiratet, zwei Hunde
	\item Geboren in Wanne-Eickel
	\item Wohnhaft im Grüt (2008 in die Schweiz ausgewandert)
	\item Senior Unix Systems Administrator in Zürich
	\item Bekennender ">Alter Sack"<
	\item Mehr unter \url{http://dirk.deimeke.net/} \\
		Podcast \url{http://deimhart.net/}
\end{itemize}
}

\frame{\frametitle{Dirk Deimeke (Ubuntu)}
\begin{itemize}
	\item Vereinsvorsitzender ubuntu Deutschland e. V.
	\item Offizieller Ansprechpartner für das deutsche LoCo-Team
	\item Mitglied im schweizer LoCo-Team
	\item Ubuntu Member
	\item Organisationsteam der Ubucon
	\item ... mehr unter \url{http://d5e.org/ubuntu}
\end{itemize}
}


\section{Einführung}
\frame{\frametitle{Einführung}
\begin{block}{Bin ich hier richtig?}
\begin{itemize}
        \item Das wird sich gleich zeigen ... :-) \pause 
\end{itemize}
\end{block}

\begin{block}{Worum geht es?}
\begin{itemize}
	\item Um Wikis im Allgemeinen
	\item Um MediaWiki und Trac im Besonderen \pause 
\end{itemize}
\end{block}

\begin{block}{Worum geht es hier nicht?}
\begin{itemize}
        \item Um eine Marktübersicht vorhandener Wikis.
\end{itemize}
\end{block}
}

\frame{\frametitle{Hinweis}
\begin{itemize}
        \item Die Session soll kein sturer Vortrag, sondern vielmehr ein Miteinander sein.
        \item Bei Fragen bitte immer fragen!
\end{itemize}
}

\section{Warum Wiki?} 
\frame{\frametitle{Weshalb MediaWiki?}
\begin{block}{Vorteile}
\begin{itemize}
        \item MediaWikis sind schnell und einfach installiert. \pause
        \item Sie laufen auf (fast) jeder Plattform, sogar auf Windows ;-) \pause
        \item Integrierte Versionierung \pause
        \item Da Wikipedia das MediaWiki nutzt, ist es sehr bekannt. \pause
        \item Es gibt sehr viele Erweiterungen (aber nicht alle tun das, was sie sollten!). \pause
        \item Einfache Rechteverwaltung (leider bloss siteweise einfach) \pause
        \item Einfache Datensicherung
\end{itemize}
\end{block}

}

\frame{\frametitle{Beispiele von MediaWiki}
\begin{itemize}
        \item Wikipedia \url{http://wikipedia.de} \pause
        \item Jedipedia \url{http://jedipedia.de} \pause
        \item Semantic MediaWiki \url{http://semantic-mediawiki.org/} \pause
        \item Web of Life Wiki \url{http://web-of-life.org.uk/wiki} \pause
        \item Nintendo-Wiki \url{http://nintendo.wikia.com/wiki/Nintendo_Wiki} \pause
        \item Pac-Man-Wiki \url{http://pacman.wikia.com/wiki/Pac-Man_Wiki} \pause
        \item Wiki-Index \url{http://wikiindex.org/}
\end{itemize}

}
% Warum Trac
\frame{\frametitle{Warum Trac?}
\begin{block}{Vorteile}
\begin{itemize}
	\item Trac vereint ein Wiki, ein Ticket-System und einen Browser für Versionsmanagement-Systeme (Subversion, Bazaar, Git, Mercurial, ...) unter einer ">Haube"<. \pause 
        \item Daher eignen sich Tracs besonders für Projektumgebungen. \pause
        \item Es gibt eine sehr ausgeklügelte Rechteverwaltung. \pause
	\item Trac ist sehr gut (englischsprachig) dokumentiert. \pause
	\item Es gibt viele Plugins. \pause
\end{itemize}
\end{block}
}

\frame{\frametitle{Beispiele für Trac-Installationen?}
\begin{itemize}
	\item Übersicht \url{http://trac.edgewall.org/wiki/TracUsers}
	\item Sourceforge \url{http://sourceforge.net/apps/trac/sourceforge/}
	\item ubuntuusers.de Webteam \url{http://trac.ubuntuusers.de/}
	\item Munin \url{http://munin-monitoring.org/}
	\item Gobby (sehr hübsch!) \url{http://gobby.0x539.de/trac/}
	\item Übersetzung des Subversion Buchs \url{http://www.svnbook.de/}
	\item DeimHart (nicht öffentlich)
\end{itemize}

}


\section{Einrichtung}
\frame{\frametitle{Vorbedingungen}
\begin{block}{Webserver}
\begin{itemize}
        \item Wir gehen von einem Apache-Webserver aus, der aus den Repositories gezogen wird.  \pause
        \item \texttt{sudo apt-get install apache2} \pause
\end{itemize}
\end{block}
\end{itemize}
}
\frame{\frametitle{Vorbedingungen für Trac}
\begin{itemize}
	\item Python ab 2.4, 3.0 und grösser wird noch nicht unterstützt. \pause
	\item setuptools \pause
	\item Genshi \pause
	\item Datenbank
	\begin{itemize}
		\item SQLite ist in Python ab Version 2.5 enthalten 
	\end{itemize}
\end{itemize}
}

\frame{\frametitle{Optional für Trac}
\begin{itemize}
	\item Trac kann Standalone ohne weiteren Server laufen \pause
	\item ">richtiger"< Webserver
	\begin{itemize}
		\item Apache 2
		\item andere möglich, aber nicht dokumentiert 
	\end{itemize} \pause
	\item ">richtige"< Datenbank
	\begin{itemize}
		\item MySQL
		\item PostgreSQL 
	\end{itemize} \pause
	\item Versionskontrollsystem
	\begin{itemize}
		\item Subversion
		\item Andere (Bazaar, Git, Mercurial, ...)
	\end{itemize} \pause
\end{itemize}
}

\frame{\frametitle{Installation Trac (1)}
\begin{exampleblock}{Einschränkung für diesen Vortrag}
Wir werden hier Trac mit Apache als Webserver und Subversion als Versionsverwaltungsssystem und Authentifizierung über den Webserver installieren.

Es gibt eine grosse Menge an weiteren Möglichkeiten, die aber hier den Rahmen sprengen würden. \pause
\end{exampleblock}

\begin{block}{Überprüfung der Vorbedingungen}
\begin{itemize}
	\item \texttt{aptitude show apache2}
	\item \texttt{aptitude show libapache2-python}
	\item \texttt{aptitude show python-pysqlite1.1 python-pysqlite2}
	\item \texttt{aptitude show python-setuptools}
\end{itemize}
\end{block}
}


\section{Live Demonstration}

\frame{\frametitle{Live Demonstration}
\begin{itemize}
        \item MediaWiki \pause
        \item Trac 
\end{itemize}
}

\section{Backup}
\frame{\frametitle{Backup MediaWiki}
\begin{block}{Sicherung}
\begin{itemize}
        \item Das MediaWiki ist sehr einfach zu sichern. Es müssen bloss zwei Bereich gesichert werden.  \pause
        \item Das Verzeichnis, in dem das Wiki liegt. \texttt{/var/www/wiki} \pause
        \item Ein Dump der Datenbank erzeugen \texttt{mysqldump -u [username] -p [password] [databasename] > [backupfile.sql]}
        \item Die Sicherung lässt sich leicht skripten. 
\end{itemize}
\end{block}

}
% Backup Trac

\frame{\frametitle{Backup Trac}
\begin{block}{Sicherung von Trac leider etwas komplexer}
\begin{itemize}
        \item Datei-Sicherung
	\begin{itemize}
		\item Trac-Konfigurationsdatei \texttt{/pfad/zum/trac/conf/trac.ini}
		\item Dateianhänge des Wikis \texttt{/pfad/zum/trac/attachments}
	\end{itemize} \pause
        \item Trac-Daten, \\ \texttt{trac-admin hotcopy /pfad/zum/trac trac.backup}. \pause
        \item Eventuell Daten der Versionsverwaltung, \\ \texttt{svnadmin dump /pfad/zum/svn > subversion.backup}. \pause
        \item Das ganze läuft über ein selbstgeschriebenes Skript.
\end{itemize}
\end{block}

}


\end{document}
